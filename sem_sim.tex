%%%%%%%%%%%%%%%%%%%%%%% file typeinst.tex %%%%%%%%%%%%%%%%%%%%%%%%%
%
% This is the LaTeX source for the instructions to authors using
% the LaTeX document class 'llncs.cls' for contributions to
% the Lecture Notes in Computer Sciences series.
% http://www.springer.com/lncs       Springer Heidelberg 2006/05/04
%
% It may be used as a template for your own input - copy it
% to a new file with a new name and use it as the basis
% for your article.
%
% NB: the document class 'llncs' has its own and detailed documentation, see
% ftp://ftp.springer.de/data/pubftp/pub/tex/latex/llncs/latex2e/llncsdoc.pdf
%
%%%%%%%%%%%%%%%%%%%%%%%%%%%%%%%%%%%%%%%%%%%%%%%%%%%%%%%%%%%%%%%%%%%


\documentclass[runningheads,a4paper]{llncs}

\usepackage{amssymb}
\setcounter{tocdepth}{3}
\usepackage{graphicx}
\usepackage{graphics}
\usepackage{multirow}
\usepackage{color}
\usepackage{paralist}
\usepackage{url}
\usepackage{subfigure}
\usepackage{listings} 
\lstloadlanguages{html,xml}
\definecolor{grey}{rgb}{0.9,0.9,0.9} 
\lstset{
        tabsize=2, 
        frame=single, 
        breaklines=true, 
        basicstyle=\footnotesize\ttfamily,
        backgroundcolor=\color{grey},
        xleftmargin=0mm,
        xrightmargin=0mm,
        captionpos=b
}

\usepackage{url}
\urldef{\mailsa}\path|{pavan, pramod, amit}@knoesis.org|    
\newcommand{\keywords}[1]{\par\addvspace\baselineskip
\noindent\keywordname\enspace\ignorespaces#1}

\graphicspath{./images/}
\DeclareGraphicsExtensions{.pdf,.jpeg,.png}	

\usepackage{color} 
\definecolor{gray}{rgb}{0.7,0.7,0.7}
\definecolor{blue}{rgb}{0.1,0.1,0.5}
\newcommand{\attention}[1]{{\color{red}\textbf{#1}}}
%\newcommand{\comment}[1]{\textit{\color{blue}#1}}
%\newcommand{\comment}[2]{{\color{blue}\textit{\textbf{#1}}(\textit{#2})}}
\newcommand{\comment}[2]{}

%\newcommand{\softdelete}[1]{{\color{gray}\textit{#1}}}
\newcommand{\softdelete}[1]{}

\newcommand{\uri}[1]{\texttt{#1}}
\newcommand{\literal}[1]{`\textit{#1}'}
\newcommand{\BibTeX}{{\sc Bib}\TeX}
\lstset{basicstyle=\small}
\begin{document}

\mainmatter  % start of an individual contribution

% first the title is needed
\title{Thesis Title \thanks{.}}

% a short form should be given in case it is too long for the running head
\titlerunning{Following Dynamic Events by Detecting Semantically Related Hashtags}

% the name(s) of the author(s) follow(s) next
%
% NB: Chinese authors should write their first names(s) in front of
% their surnames. This ensures that the names appear correctly in
% the running heads and the author index.
%
%\author{Pavan Kapanipathi\and Julia Anaya\and Alexandre Passant}
%
% (feature abused for this document to repeat the title also on left hand pages)

% the affiliations are given next; don't give your e-mail address
% unless you accept that it will be published
%\author{Pavan Kapanipathi\and Prmod Koneru\and Amit Sheth}
\author{}
\authorrunning{P.Koneru}

\institute{
Kno.e.sis Center, CSE Department\\
Wright State University, Dayton, OH - USA \\
{\textbraceleft pramod \textbraceright@knoesis.org}
}
%\mailsa\\
%\url{http://www.deri.ie/}}

%
% NB: a more complex sample for affiliations and the mapping to the
% corresponding authors can be found in the file "llncs.dem"
% (search for the string "\mainmatter" where a contribution starts).
% "llncs.dem" accompanies the document class "llncs.cls".
%

%\toctitle{Semantic Multicasting}
\tocauthor{P. Koneru}
\maketitle

\begin{abstract}
In this era of big-data, collecting relevant data is the major first step towards analyzing, let alone get some information of it. There has been a lot of research on analysis part of it yet a very little is done in the data collection part. Traditional way of collecting/crawling data, from social networking sites say Twitter, include manually coming up with the keywords for an event and start crawling. This has its own dis-advantages
\newline
\newline 1) It is very difficult to come up with all the relevant keywords for an event manually. Also the keywords for an event are platform dependent (For example the keywords used in twitter may be totally different from that of G+ or Facebook). So we need a more efficient way.
\newline
\newline 2) Even though if we can come up with keywords, data on web changes every minute, events evolve over time and people may not be using the same keywords as before in-fact they may be using new keywords. Manually keeping up with this is very difficult rather say not possible.
\newline
\newline So there is a necessity of an efficient and automatic way of doing this. In this thesis, we develop a mechanism by which the system, given an event, will automatically get the keywords for crawling and also updates the keywords as the event evolves. Our algorithm uses both Semantic Web and Information Retrieval techniques to come up with the keywords and to dynamically/automatically update the keywords as the event evolves over time. Due to the informal nature of the content in social media, using IR techniques alone won't give us correct results, so we combine them with Semantic Web Techniques (using Wikipedia and DBPedia to get semantically related entities) to extract the correct keywords. And to keep up with the evolution of the event we track revisions of Wikipedia page, updating the keywords when there is a change in the Wiki page. Finally we evaluate our system with ......

\newline 
\keywords{Semantic Web, Social Web, Dynamic Events, Twitter}
\end{abstract}
\section{Introduction}
\label{sec:Introduction}	

%\input{background.tex}
%\input{architecture.tex}
%\input{implementation.tex} 
%\input{related.tex}
%\input{conclusion.tex}


\bibliographystyle{plain}
\bibliography{sem_sim}
\end{document}
