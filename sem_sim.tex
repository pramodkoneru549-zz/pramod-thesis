%%%%%%%%%%%%%%%%%%%%%%% file typeinst.tex %%%%%%%%%%%%%%%%%%%%%%%%%
%
% This is the LaTeX source for the instructions to authors using
% the LaTeX document class 'llncs.cls' for contributions to
% the Lecture Notes in Computer Sciences series.
% http://www.springer.com/lncs       Springer Heidelberg 2006/05/04
%
% It may be used as a template for your own input - copy it
% to a new file with a new name and use it as the basis
% for your article.
%
% NB: the document class 'llncs' has its own and detailed documentation, see
% ftp://ftp.springer.de/data/pubftp/pub/tex/latex/llncs/latex2e/llncsdoc.pdf
%
%%%%%%%%%%%%%%%%%%%%%%%%%%%%%%%%%%%%%%%%%%%%%%%%%%%%%%%%%%%%%%%%%%%


\documentclass[runningheads,a4paper]{llncs}

\usepackage{amssymb}
\setcounter{tocdepth}{3}
\usepackage{graphicx}
\usepackage{graphics}
\usepackage{multirow}
\usepackage{color}
\usepackage{paralist}
\usepackage{url}
\usepackage{subfigure}
\usepackage{listings} 
\lstloadlanguages{html,xml}
\definecolor{grey}{rgb}{0.9,0.9,0.9} 
\lstset{
        tabsize=2, 
        frame=single, 
        breaklines=true, 
        basicstyle=\footnotesize\ttfamily,
        backgroundcolor=\color{grey},
        xleftmargin=0mm,
        xrightmargin=0mm,
        captionpos=b
}

\usepackage{url}
\urldef{\mailsa}\path|{pavan, pramod, amit}@knoesis.org|    
\newcommand{\keywords}[1]{\par\addvspace\baselineskip
\noindent\keywordname\enspace\ignorespaces#1}

\graphicspath{./images/}
\DeclareGraphicsExtensions{.pdf,.jpeg,.png}	

\usepackage{color} 
\definecolor{gray}{rgb}{0.7,0.7,0.7}
\definecolor{blue}{rgb}{0.1,0.1,0.5}
\newcommand{\attention}[1]{{\color{red}\textbf{#1}}}
%\newcommand{\comment}[1]{\textit{\color{blue}#1}}
%\newcommand{\comment}[2]{{\color{blue}\textit{\textbf{#1}}(\textit{#2})}}
\newcommand{\comment}[2]{}

%\newcommand{\softdelete}[1]{{\color{gray}\textit{#1}}}
\newcommand{\softdelete}[1]{}

\newcommand{\uri}[1]{\texttt{#1}}
\newcommand{\literal}[1]{`\textit{#1}'}
\newcommand{\BibTeX}{{\sc Bib}\TeX}
\lstset{basicstyle=\small}
\begin{document}

\mainmatter  % start of an individual contribution

% first the title is needed
\title{Thesis Title \thanks{.}}

% a short form should be given in case it is too long for the running head
\titlerunning{Following Dynamic Events by Detecting Semantically Related Hashtags}

% the name(s) of the author(s) follow(s) next
%
% NB: Chinese authors should write their first names(s) in front of
% their surnames. This ensures that the names appear correctly in
% the running heads and the author index.
%
%\author{Pavan Kapanipathi\and Julia Anaya\and Alexandre Passant}
%
% (feature abused for this document to repeat the title also on left hand pages)

% the affiliations are given next; don't give your e-mail address
% unless you accept that it will be published
%\author{Pavan Kapanipathi\and Prmod Koneru\and Amit Sheth}
\author{}
\authorrunning{P.Koneru}

\institute{
Kno.e.sis Center, CSE Department\\
Wright State University, Dayton, OH - USA \\
{\textbraceleft pramod \textbraceright@knoesis.org}
}
%\mailsa\\
%\url{http://www.deri.ie/}}

%
% NB: a more complex sample for affiliations and the mapping to the
% corresponding authors can be found in the file "llncs.dem"
% (search for the string "\mainmatter" where a contribution starts).
% "llncs.dem" accompanies the document class "llncs.cls".
%

%\toctitle{Semantic Multicasting}
\tocauthor{P. Koneru}
\maketitle

\begin{abstract}
In this era of big-date, collecting relevant data is the major first step towards analyzing, let alone get some information of it. There has been a lot of research on analysis part of it but a very little effort has been put on data collection. Also we are in a pace where data on web changes every minute, it has become a necessity rather a mandatory thing to capture that new information. In this paper, we propose a automatic 
\keywords{Semantic Web, Social Web, Dynamic Events, Twitter}
\end{abstract}
\input{introduction.tex}
%\input{background.tex}
%\input{architecture.tex}
%\input{implementation.tex} 
%\input{related.tex}
%\input{conclusion.tex}


\bibliographystyle{plain}
\bibliography{sem_sim}
\end{document}
